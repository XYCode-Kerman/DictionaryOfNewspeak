\documentclass[lang=cn, color=black]{elegantbook}
\usepackage{multicol}

\title{新话词典}
\subtitle{战争即和平,自由即奴役、无知即力量}
\date{1964年1月1日}
\author{记司}
\institute{真部}
\version{一版}
\cover{cover.jpg}

\newenvironment{word}[2]{
    \noindent \textit{#1} \quad \textbf{#2} \\
    \indent
}{}

\newcommand{\os}{\textit{Old Speak}}

\newcommand{\eos}[1]{等同于 \os 中的 #1}

\newcommand{\vocab}{\textit{v.}}
\newcommand{\noun}{\textit{n.}}
\newcommand{\adj}{\textit{adj.}}
\newcommand{\adv}{\textit{adv.}}
\newcommand{\anyword}{\textit{anyword.}}

\newcommand{\ingsoc}{英格兰社会主义党}
\newcommand{\bb}{老大哥(\textit{Big Brother})}
\newcommand{\oceania}{大洋国(\textit{Oceania})}
\newcommand{\gold}{果尔德施坦因\footnote{党的叛徒!革命的背叛者!}(\textit{Goldstein})}

\begin{document}
    \maketitle

    \tableofcontents

    \part*{序言}
        \chapter*{政治声明}
            \section*{In Chinese}
                \textbf{本项目是小说《一九八四》的同人作品,与现实无关!}

                《一九八四》(Nineteen Eighty-Four)是英国作家乔治·奥威尔于1949年发表的长篇政治小说。

                此书在中华人民共和国、美利坚合众国、大不列颠及北爱尔兰联合王国、法兰西共和国、俄罗斯联邦及多个国家和地区均已合法出版。如果您的所在的国家或地区不允许《一九八四》的出版,则您应该不应当查看本页面。

                本人反对法西斯主义,帝国主义,霸权主义,拥护党和国家的方针政策,坚决反对切分裂国家行为自觉遵守严格落实《网络信息内容生态治理规定》以营造清朗的网络空间,建设良好的网络生态为目标反对法西斯主义,帝国主义,霸权主义,弘扬社会主义精神。

                请勿在Issue、PR区发布淫秽色情,暴力血腥谩骂侮辱,攻击他人,极端反动等过激言论,一旦出现违规问题将会由相关部门依据规定以及国家法律法规进行严格处理。
                
                \begin{center}
                    \large
                    \textbf{善语结善缘,恶语伤人心,请三思而后行。}
                \end{center}
            
            \section*{In English}
                \textbf{This project is a companion piece to the novel Nineteen Eighty-Four and has nothing to do with reality!}

                \textit{Nineteen Eighty-Four} is a full-length political novel published in 1949 by British author George Orwell.

                It has been legally published in the PRC, the USA, the UK, the French, the Russian Federation, and many other countries and territories. If your country or region does not allow publication of \textit{Nineteen Eighty-Four}, \textbf{you should not view this project.}

                I am against fascism, imperialism and hegemony.

                Please do not post obscene pornography, violent bloody abuse, insults, attacks on others, extreme reactionary and other radical statements in Issues and PR, once a violation will be dealt with in accordance with the People's Republic of China and the United States Federal and relevant laws and regulations in your region.
    
        \chapter*{版权声明}
            \begin{center}
                \large
                \textbf{本项目采用}\textit{CC0 1.0 DEED}\textbf{协议}。
            \end{center}

            在作品上适用该文本的人已经将作品 贡献 至公共领域,放弃所有他/她在全世界范围内基于版权法对作品享有的权利至法律允许的范围,包括所有类似和邻接权利。

            您可以复制、修改、发行和表演本作品,甚至可用于商业性目的,都无需要求同意。请看\textbf{下方说明。}

            除非另有明确声明,本协议项下的作品关联人,在可适用法律所允许的最大限度内,不对本作品提供担保,不承担因本作品使用产生的责任。

            当使用或引用本作品,您不得暗示\textbf{作者或声明人认可你}。

            本项目作者从未将其用于盈利目的,其中关于《1984》小说、电影等的文字、图像引用均符合《中华人民共和国著作权法》和《美国联邦版权法》中的“适当引用”。如果您认为本项目侵犯了您的合法权利,请发送邮件到xycode-xyc@outlook.com,我方将及时进行处理。
        
    \part{正文}
        \chapter{单词}
            \section{A类词}
                \textit{
                    组成A类词的,是日常事务中必须使用的那些词,比如饮食、工作、穿衣、上楼、下楼、驾车、种花、烹饪等等。它几乎全部都是我们已经使用过的那些词,如打、跑、狗、树、糖、房屋、田野等等,但与今天我们所用的英语相比,它的数目非常之小,而意义又受到严格得多的限定。凡是意思上模棱两可、有细微差别的地方,都一概删除。倘若这样,那么一个A类词就只是一种独立的喉音,表示我们都明白知道的一个概念。这样,想把A类词用到文学上,用到政治和哲学的讨论中,就完全不可能了。它所要表达的,只是简单、有明确意图的思想,通常都是涉及具体物体或者身体的动作。
                }

                \rightline{—— 乔治·奥威尔《1984》}

                新话的第一大语法是不同词性的词几乎可以完全混用。它的任意一个词,既可以用作动词,也可以用作名词、形容词或者副词,从原则来说,即使像“if”(如果)“when”(当……的时候)这样非常抽象的名词也不例外。

                此外,每个词只要加上前缀“un”就可以表示否定,加上前缀“plus”就表示强调,进一步强调就加“doubleplus”,原则上新话每一个词都是这样。

                \begin{multicols}{2}
                    \begin{word}{when}{当 \dots 时}
                        \eos{if}
                    \end{word}
    
                    \begin{word}{think}{思想}
                        \eos{thought}
                    \end{word}
    
                    \begin{word}{knife}{切;伤到}
                        \eos{cut}
                    \end{word}
    
                    \begin{word}{\vocab + ful}{}
                        将 \os 中的动词变化为\adj
                    \end{word}

                    \begin{word}{\vocab + wise}{}
                        将 \os 中的动词变化为\adv
                    \end{word}

                    \begin{word}{un + \anyword}{}
                        表达否定。
                        
                        例如 \os 中的 warm 可表达为 uncold
                    \end{word}

                    \begin{word}{plus + \anyword}{}
                        表达强调。
                        
                        例如 \os 中的 very good 可表达为 double plus good
                    \end{word}
                \end{multicols}

                新话语法的第二个显著特征是它的规则性。几乎所有词形的变化都遵循同样的规则。这样,每个动词的过去式和过去分词就都是以“ed”结尾,“steal”的过去式就成了“stealed”,“think”的过去式是“thinked”,新话中其他的词也是一样。
\end{document}